\chapter{Besoins et objectifs}
Les objectifs du projet portent sur plusieurs volets. Premièrement, il est nécessaire de considérer l'aspect de paramétrisation des simulations et de dimensionnement des composantes employées selon un cahier des charges précis. En ce qui attrait à l'outil de dimensionnement à proprement parler l'outil de paramétrisation et de contrôle, consiste à fournir un outil de contrôle vérifier le dimensionnement des composantes, qui devra être calculé de manière théorique et approximative. Les objectifs à cet endroit sont de:

\begin{itemize}
\item Fournir un outil de dimensionnement pour chaque plateforme qui soit convivial et facile d'utilisation;
\item Livrer un outil de dimensionnement pour chaque plateforme qui utilise les paramètres usuels de nomenclature utilisés dans les simulateurs.
\end{itemize}

\paragraph{}Deuxièmement, pour ce qui est des simulateurs pour chacune des plateforme, il se doivent de:
\begin{itemize}
\item Valider que la conception choisie est fonctionnelle;
\item Permettre la comparaison des résultats de simulation pour différents paramètres de dimensionnement.
\end{itemize}
Les plateformes sur lesquelles les simulations doivent être livrées sont Simulink(SimPowerSystems), PSIM et Opral-RT. SPS est un outil de simulation générique qui permet de simuler tout type de circuits, toutefois ce côté générique cause des temps de simulation généralement plus longs pour une même précision comparé à des simulateurs spécifiques comme PSIM. L'usage de SPS est problématique au niveau des variations rapides. PSIM est spécialement conçu pour les circuits d'électronique de puissance et de contrôles de moteur, tandis que les simulateurs génériques sont conçus pour les circuits électriques de base. Cet outil permet une meilleure rapidité et une meilleure précision. Par ailleurs, il est plus robuste aux variations rapides. Le simulateur OPA-4500 de Opal-RT est un simulateur temps réel qui permet une comparaison directe des résultats avec SPS. Les pas de simulations sont généralement faible, compte tenu de l'optimisation effectuée et de la puissance des composantes. Cet outil permet de réaliser des simulations en temps réel à partir de SPS. Par ailleurs, l'utilisation d'un simulateur temps réel présente la possibilité de tests d'intégration en temps réel

\paragraph{}Troisièmement, en ce qui attrait à la documentation technique, elle présente d'une part les résultats de calculs de dimensionnement théoriques, d'autres part un guide d'utilisation des simulations et de l'outil de contrôle. L'objectif de cette documentation est donc de:
\begin{itemize}
\item Présenter des exemples d'utilisation des simulations et de l'outil de contrôle et de dimensionnement.
\end{itemize}
\paragraph{}La validation croisée des simulations est nécessaire afin de juger de la validité des résultats, à cet effet, il est donc nécessaire d':
\begin{itemize}
\item Implanter une validation croisées des 3 simulateurs.
\end{itemize}