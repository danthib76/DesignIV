%!TEX root = ../rapport.tex
%!TEX encoding = UTF-8 Unicode
\chapter{Outil de contrôle et de dimensionnement}
Cette section présente l'interface de contrôle et de dimensionnement des simulateurs PSim et SPS.

En premier lieu, on doit indiquer à l'outil de contrôle les différentes simulations ainsi que les chemins d'accès respectifs pour y accéder. La figure \ref{outil1} présente les informations nécessaires pour le bon fonctionnement de l'outil.

 \begin{figure}[htb]
 \centering
 \makebox[\textwidth][c]{\includegraphics[width=0.95\textwidth]{fig/outil1.png}}
 \caption{Préprogrommation de l'outil de contrôle et de dimensionnement}
 \label{outil1}
 \end{figure}

Une fois les informations des simulations mises en place, il suffit de choisir la simulation à lancer, de choisir le simulateur cible et d'entrer les différents paramètres de simulation. La figure \ref{outil2} présente l'interface de lancement et de programmation des paramètres.

 \begin{figure}[htb]
 \centering
 \makebox[\textwidth][c]{\includegraphics[width=0.95\textwidth]{fig/outil2.png}}
 \caption{Interface de lancement et de programmation des paramètres}
 \label{outil2}
 \end{figure}

 Cet outil permet de gagner un temps précieux lors de l'enchaînement rapide des simulations, car tous les paramètres capitaux sont regroupés au même endroit. De plus, comme l'outil permet de contrôler les deux plateformes de simulation, les valeurs vont être les mêmes d'une plateforme vers l'autre. On s'assure ainsi d'une homogénéité des résultats et on évite des erreurs humaines liées à la manipulation séquentielle des blocs dans les plateformes. 
